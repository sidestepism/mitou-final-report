\section{今後の課題,展望}
%開発後に残っている課題とその課題に対する今後の取り組み、および開発成果の普及や実用化に関する予定や計画について記述してください。
\subsection{今後の課題}
\subsubsection{ソフトウェアとしての課題}
現在のシステムは,理想的な環境下では10~20Hzで動作するが,カメラとPCの通信のハードウェア的な相性の問題で,ポイントクラウドの取得に時間がかかり,これが処理を律速している.さらに,室内でのユーザスタディの先に,屋外での使用を考えた場合は,赤外光を使用する現在の測距法では対応できない.そのため,深度情報を取得するための様式の変更をセンサの選定を含めて行うことでより実環境できるシステムを作成できると考えられる.

さらに,開発物の生成する音のソースとして楽器を用いているが,どのアフォーダンス情報をどの楽器に割り当てるのか,さらに,面の角度といった情報をどの音パラメータに割り当てるかといった手法は決め打ちであり,詳細な検討はなされていない.さらなるユーザスタディを通じて,この点を改良する.
\subsubsection{ハードウェアとしての課題}
センサを小型化するとともに,通信部分を無線に切り替える手法を開発することで,計算機を持ち運ばずに,ゴーグルの着用のみによってSightを装着することが期待できる.この場合,計算機としてノートPCを使用する必要がないため,通信部分を律速としてシステムを動作させることが期待できる.
\subsubsection{プロジェクトとしての課題}
先述したように,最終形としての本デバイスの着用による新しい視覚様式の体験はまだ限られた数の人にしか試してもらっていない.さらに多くの人を対象とした意見の収集が必要である.さらに,長時間の着用による,知覚様式の時間変化についても,また詳細な調査ができていないため,今後のユーザスタディが期待できる.

\subsection{今後の展望}
我々はプロジェクト開発期間を通じて,新しい視覚様式の獲得のために必要なものを吟味し,開発と改良を重ねた.その結果,目指していたデバイスを開発し,初期のユーザースタディを完了した.今後のプロジェクトの第一の目標は,さらなるユーザスタディ,展示の機会を通じて,多くの人にこのデバイスを体験してもらうことである.そのために,モバイルアプリケーションとしての実装や,複数デバイスの制作,モバイルアプリケーションとしての実装などよいパッケージングを行い,常設展示やデモの機会を得ることで,より多くの方にSightの体験をしてもらったり,体験について伝える機会を設けるための努力を行っていきたいと考えている.

併せて,広報の機会を通じて,実用的なデバイスとしての新しい可能性を企業との連携あるいはユーザから得られた意見をもとにして探ってゆきたいと考えている.