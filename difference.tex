\section{実施計画書内容との相違点}
%本プロジェクト終了後の開発結果について、実施計画書に記載の開発内容と比較して、計画より進んだ部分、計画通りに進まなかった部分、あるいは変更箇所について記述してください。
%実施計画書と相違がない場合は、『実施計画書内容と相違なし』と記載してください。
実施計画書では,「カメラをとヘッドフォンを用い、カメラを通して得られる空間の情報を神経科学の知見に基づき音に変換し提示することで、人間の空間感覚を変容させ、進化させることを目指したHuman Augmentation Deviceを開発する。」とあった,開発したデバイスの音変換アルゴリズムには神経科学的な知見は反映されていないという点で異なる.また,具体的な実装の箇所について,「自然の環境音を損なわないようにするために、マイクロフォンを介して音声を取得し、生成された音に重畳する。」と記載した点については,開発したデバイスでは自然音を重畳することはしなかったため,異なる.

この2点は実施計画書からの変更点であるが,これは第一の目的とする「人間の空間感覚を変容させる」ためのデバイス開発を目指す上での,改良及び評価,考察の結果修正されたものである.まず,バージョン1の評価によって,低次視覚野の情報を音に直接変換するという神経科学的な方法は情報が複雑すぎるため難しいということがわかった.さらに,新しい視覚実現のために,我々は通常の感覚器官に頼らない新しい情報リソースを脳に入力することによってこれを実現する設計を考えた.そのため通常の環境音の重畳は必ずしも必要ではなくなった.