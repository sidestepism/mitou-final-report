\section{背景及び目的}
%本プロジェクトを実施する以前の状況・背景とプロジェクト実施の目的とを記述し、背景にある問題をどのように解決あるいは変化させたいかを明らかにしてください。
\subsection{背景}
\subsubsection{視覚の役割と性質}
人間をはじめとする多くの動物は,自身が置かれた環境の空間情報を,主に視覚によって認識していると考えられる.晴眼者の目を突然塞いだ時,容易には歩けず,物体や壁に転倒することから,上記のことは客観的に疑いようがないといえる.一方で「視覚=見える」という経験自体は主観的な体験であるため,目の前の人間がどういった「見え」の経験をしているかを知ることは一般に難しいものである.この報告書では,視覚とは「空間の情報に関する知覚」のことを指し,以後「見え」という言葉と同一のものとして扱う.


\subsubsection{視覚の多様性}
ここで,視覚を生成する脳は環境からの入力の変化に柔軟に対応できるような可塑性,柔軟性を備えている.そのため,脳の柔軟な機能変化が亢進した脳において,視覚という知覚経験は必ずしも通常の眼球を使用して実現されるものではないことが,先行事例によって示唆されている.例えば後天的な視覚障害者で,杖や足音の反響音,周囲の環境音の変化から,壁や人の流れを感知して自由に歩き回ったりすることができるという事例は多く報告されている.彼らの脳の働きを調べると,音を聞いている時に特に視覚を処理する脳領域が活動していることが明らかになっている.彼らは晴眼者が普段目を使って物を見るように,「耳を使って見る」という新しい「見え」の形を手に入れたと考えられる.他の例としては,世界が上下反転して見える「逆さメガネ」をかけて1週間暮らした心理学者の事例がよく知られている.上下反転した視野に適応し,普段通り生活をすることができるようになることは驚異的である.人間以外の例も存在する.例えばマウスの視覚を処理する脳領域に磁場の情報を入力すると,地磁気が見えるかのように動き回れるようになったといったことが報告されている.

すなわち,我々は脳の機能的な可塑性あるいは柔軟性によって,脳に入力される様々な情報を編集・統合して「視覚=見え」という経験を実現していることがわかる.視覚には多様な様式がある.

\subsubsection{問題点}
これらの報告を整理すると,「視覚=見え」には多様性があることが知られているが,見えが主観的な体験であるという性質上,これまでは個々人によって多様であるはずの視覚体験について我々が知ることが困難であるという問題があると考えることができる.視覚の多様性を理解することは,大きく分けて次のふたつの社会的意義があると考える.第一に,脳がある「見え」から別の「見え」を獲得するプロセスを理解することによって,脳の可塑性についての新しい知見を得ることが期待できるという点である.例えば新しい見えを脳が獲得するプロセスを加速する方法が明らかになれば,視覚障害者の支援といった福祉的な応用に繋がると考えられる.脳の機能は明らかにされていないことが多いため,探求の余地がある問題である.第二に,見えが複数存在するということを理解することによって,異なる視覚様式を持つ他者の理解に繋がるということが考えられる.例えば,通常異なる知覚様式を持つ人々が同じ視覚様式の中で作業する新しい活動(スポーツ,絵画鑑賞)を考えることができる可能性がある.

また,見えの多様性を理解するためのデバイスを開発するためにIT技術を利用することには,以下のようなふたつの利点がある.第一に,どのようにすれば見えを実現するべきかという様式が明らかにされていないため,様々な情報変換の手法をソフトウェアの変更によって試すことができるという点である.第二に,逆さメガネのような古典的な手法に囚われない,様々なセンサを使用することができるという点である.

\subsection{目的}
我々は上記の問題点に鑑み,IT技術を駆使することによって「主観的な経験を通じて視覚の多様性を理解するデバイスの開発」を目指す.視覚の多様性を理解するためには,普段慣れ親しんだ視覚様式とは異なる視覚様式を主観的経験によって手に入れられればよいと考えられる(=空間知覚の拡張).さらに,本プロジェクトでは「眼球=光」に依らない新しい情報として,五感の中で視覚の次に環境に関する情報を伝えると考えられる聴覚が処理する「音」に着目する.すなわち,環境に関する情報をセンシングし,音に変換して提示する(=聞こえる化)ことによって,新しい視覚経験を手に入れることを目指す.

すなわち,本プロジェクト「空間知覚拡張のための``聞こえる化''デバイスの開発」実施の目的は,「環境に関する情報をセンシングし,音に変換して提示することによって新しい視覚様式を手に入れるデバイス(プロダクト名"Sight")を開発すること」である.












