\section{背景及び目的}
%本プロジェクトを実施する以前の状況・背景とプロジェクト実施の目的とを記述し、背景にある問題をどのように解決あるいは変化させたいかを明らかにしてください。
\subsection{背景}
本章では本プロジェクトの開発をはじめた動機について説明する.

本プロジェクトのメンバーは,
メンバー4名、神経科学、心理学、ユーザーインターフェイス、情報デザインと普段別々の分野で活動しているが,それぞれ各々の領域から,ヒトの知覚,とくに知覚の拡張可能性や,それを担う脳の可塑性について興味を持っていた.

私達人間がどのように世界を見ているかというのは,ある種当たり前で疑いようのないものに感じられるが,人間の脳の知覚を担うシステムは環境からの入力の変化に柔軟に対応できるような可塑性,ロバスト性を備えている.たとえば後天的な視覚障害者で,杖や足音の反響音,周囲の環境音の変化から,壁や人の流れを感知して自由に歩き回ったりすることができるという事例は多く報告されているが,かれらの脳の働きを調べると,音を聞いている時に特に視覚を処理する脳領域が活動していることが明らかになっている.彼らは、僕たちが普段目を使って物を見るように、、「耳を使って見る」という新しい「見え」の形を手に入れた、と考えられる.

ほかにも,世界が上下反転して見える「逆さメガネ」をかけて1週間暮らした心理学者の事例はよく知られている.上下反転した視野に適応し,普段通り生活をすることができるようになることは驚異的である.

このような知覚の可塑性は人間以外にも備わっている.
たとえばマウスの視覚を処理する脳領域に、磁場の情報を入力すると、地磁気が見えるかのように動き回れるようになった、といったことが知られています.

しかし,このような特殊な事例を聞いても,彼らがどのような見えを獲得したのか,その見えは具体的
にどのような体験なのかというのは明らかにならない.このような知覚の可塑性というどんな人の脳にも備わっている重要な脳の機能は,実際に体験してみないと理解することができないにもかかわらず,日常にはこのような体験をすることができる機会というのはめったにない.

自分の持っている目や耳といったあたりまえの知覚入力様式を離れ,
新しい知覚のかたちを手に入れるというエキサイティングな体験を,
私達のような知覚に興味を持って研究を進めている研究者,さらには一般の方々にも,
わかりやすいかたちで提供することは,
陰に陽にひろく人間の知覚・認知・情報処理システムへの拡張が進む現代,
必要な教養を育むために重要である.

我々はこのような経緯から,人間の可能性に満ちた「知覚の可塑性」についての研究をさらにすすめること,そしてこのような人間の知覚の拡張可能性を多くの人に伝え,エンターテインメントとして楽しんでもらうことを目指し,世界の空間情報を音に変換する"Sight"というプロダクトを開発した.

\subsection{目的}













