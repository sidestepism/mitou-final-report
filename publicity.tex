\section{社会へのインパクト}

この章では,Sightプロジェクトと社会との関わりについて説明する.


\subsection{展示}

本プロジェクトでは,開発期間を通して,東京大学制作展、デジタルコンテンツエキスポ、東京藝術大学「障がいとアーツ」プログラムなど、様々な形でSightを展示し,累計500名以上の方々に各バージョンのSightを試していただくことができた.下記に主な展示を行ったイベントの一覧を示す.

\begin{itemize}
 \item 東京大学制作展EXTRA 2015年7月10日-7月13日
 \item デジタルコンテンツEXPO2015 2015年9月
 \item 東京大学制作展EXTRA 2015年11月20日-7月23日
 \item 障害とアーツ 2015年12月5日-12月6日
\end{itemize}

これらの展示来場者の方々はごく短い時間の体験が大部分だったので,長時間の装着を通じたSightならではの感覚体験を報告が得られたのはごく一部であった.
また比較的長時間試していただいた方にも上達速度にはばらつきがあり,
数分間で壁の配置などを把握できるようなった方もいれば,もっと長時間試してもなかなか把握することができないという方もいた.

デジタルコンテンツEXPOでは,今後の開発指針の策定のためにアンケートを日本語・英語で作成・配布したところ,下記のような感想が多く集まった.

\begin{itemize}
 \item 机の壁や位置がわかった
 \item どういう仕組なのか気になった
 \item 音から周りの光景を想像することができた
 \item ドキドキした・楽しかった
\end{itemize}


\subsection{広報}

プロジェクトでは,「Sightを通して得られる体験は
実際に足を運んで装着してみないと伝わらない」ということを
早期のプロトタイピングから認識していた.
このため,本プロジェクトの目標を達成するためには
目標とした性能を発揮できるプロダクトを制作するだけではなく
より多くの人に知ってもらい,体験してもらう取り組み,
すなわち広報戦略を用意することにした.

未踏のプロジェクト期間中から,
Innovative Technologies受賞で得たデジタルコンテンツEXPO展示の機会などを
広報のために最大限に利用するために,
展示を体験したプレス関係者のために,
未踏で使っているプレゼン資料や,
記事に使える代表的な画像をまとめたプレキットを用意し,
広報関係者の方に,インターネット経由で配布を行った.

結果として,下記の媒体に露出を得ることができた.

\begin{itemize}
 \item withnews 2015年1月1日『「ハッカソンはビジコンじゃない」 東大の硬派なコンペに強者続々』
 \item gizmodo 2015年7月13日 『人間はどこまで進化できるのか? "音で見る"感覚拡張デバイス「Sight」開発者インタビュー』
 \item 攻殻機動隊REALIZE project 2015年7月27日 『2029年の義体を一部体感。「東京大学制作展エクストラ2015 グッバイ・マイ・ボディ」展参加。視覚や聴覚・感情をハックする作品で攻殻機動隊世界を検証』
 \item IGNITION 2016年1月 『New VR Headset Invites Users to “See with Their Ears”』
\end{itemize}


これらの記事を見た一般の方や広報関係者が
最終成果発表のプレゼンテーションやデモに足を運んでくださり,
次のメディア取材などにつなげることができている.
