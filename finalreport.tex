%!TEX encoding = UTF-8 Unicode
\documentclass{jsarticle}
\usepackage[dvips]{graphicx}
\usepackage{url}
\begin{document}

%図番号に節番号を付加。節番号-図番号(図番号は節ごとに付ける)。
%http://www.biwako.shiga-u.ac.jp/sensei/kumazawa/tex/011.html
 \makeatletter
    \renewcommand{\thefigure}{
    \thesection.\arabic{figure}}
    \@addtoreset{figure}{section}
  \makeatother

%表番号に節番号を付加。節番号-図番号(表番号は節ごとに付ける)。
%http://www.biwako.shiga-u.ac.jp/sensei/kumazawa/tex/011.html
  \makeatletter
    \renewcommand{\thetable}{%
    \thesection.\arabic{table}}
   \@addtoreset{table}{section}
  \makeatother

\title{最終成果報告書}
\author{和家尚希 宗像悠里 鈴木良平 伏見遼平}
\maketitle

\tableofcontents % 目次
\thispagestyle{empty} % ページ数を表示しない


\section{はじめに}
\section{開発の動機}

\section{実装の概要}
\subsection{外界の認識}
\subsubsection{バージョン1: 低次の特徴点・局所特徴量(SURF)の分布}
\subsubsection{バージョン2: 一般物体認識のDNNを用いたモデル}
\subsubsection{バージョン3: 視野内数点の深さ情報}
\subsubsection{バージョン4: 平面配置から行為可能性を解析}
\subsection{音響合成戦略}
\subsubsection{Granular Synthesisを用いた音響合成}
\subsubsection{Corpus-based Granular Synthesis (catart~) を用いた音響剛性}
\subsubsection{楽器、立体音響}

\section{評価}
\subsection{タスク}
\subsection{ユーザスタディ}
\subsection{インタビュー}

\section{社会へのインパクト}
\subsection{広報}
\subsection{展示}
\section{おわりに}
\subsection{課題}
\subsection{今後の展望}


\end{document}