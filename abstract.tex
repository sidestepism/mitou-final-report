\section{プロジェクト概要}
%本プロジェクトで\UTF{2460}どんなソフトウェアまたはシステムを開発するのか(目標とする機能・性能やその実現手段など)と、\UTF{2461}開発結果としてどこまで実現できたのかなど、開発テーマの概要となる点を『簡潔に記述』してください。
%本項はプロジェクトの目標と結果の概要をわかり易くまとめるものであり、開発内容・結果の詳細については、下記の「4. 開発内容」で記述してください。
\subsection{どんなシステムを開発するのか}
本プロジェクト実施の目的は,「環境に関する情報をセンシングし,音に変換して提示することによって新しい視覚様式を手に入れるデバイスを開発すること」であった.これの実現に目指した具体的なシステムの機能を説明する.
\subsubsection{ターゲット}
我々は本プロジェクトを通して,人間の可能性に満ちた「知覚の可塑性」についての研究をさらにすすめ,さらにこのような人間の知覚の拡張可能性を多くの人に伝え,エンターテインメントとして楽しんでもらいたいと考えた.そのため,本デバイスのターゲットは通常異なる視覚様式を持つはずの晴眼者,視覚障害者を共にターゲットにした.このことによって,自分の持っている目や耳といったあたりまえの知覚入力様式を離れ,新しい視覚様式を手に入れるという驚異的な体験を,全ての人々に提供することを可能にできると考えた.
\subsubsection{ソフトウェアの機能}
ソフトウェアの機能としての第一の要求仕様は,あらゆるセンサデータを処理した上で,音に変換することができる構成になっているということである.ただし,どのような情報変換をすれば新しい視覚を得られるかについては全くわからないため,試行錯誤による改良が必要である.第二の要求仕様は,我々の感覚器官が行っているように,リアルタイムに近い速度での情報処理が可能であることである.
\subsubsection{ハードウェアの機能}
一般に脳の可塑性によって新しい視覚様式が生まれるには,逆さメガネの知見にあるように数時間から数日間の訓練が必要になると考えられる.そのため,新しい視覚様式を手に入れるまでの間,安全につけられるようなハードウェアを考える必要がある.さらに,ターゲットが晴眼者も含んでいることから,着用することで通常の光を通じた視覚経験ができなくなるようなデァインになっていることが望ましいと考えられる.
\subsubsection{成果物の使用によって得られるべき体験}
まずは実際に開発者が被験者となり,知覚様式の変化を体験できれば,本デバイスの着用によって新しい知覚様式を手にに入れることができたと考えられる.ここで,視覚は主観によってしか体験できないものであるため,主観報告によってどのような体験ができたかを報告することが必要である.

さらに,本システムのターゲットがあらゆる視覚様式を持つ人として設定したため,健常者及び,視覚障害者の使用による主観経験の報告を取得する必要がある.

\subsection{開発結果としてどこまで実現できたか}
本プロジェクトでは,「デバイスの試作」,「展示による体験のフィードをバック取得」,「それによるデバイスの改良」を重ねた.その結果,視覚の生成には「自分がどういう行動(e.g. 前進する,接触する,座る)をその空間で引き起こせるかという情報」が重要であるということが示唆された.また,聞き心地の良い音でないと,長時間の着用ができないことが示唆された.そのため,改良を重ねた最終形として,深度カメラから空間探索行動にとって重要な情報である壁,床,ある程度の大きさを持つ物体等の存在を検出し,それをバイオリン等の楽器音に割り当て,立体音響を生成するソフトウェアを開発した.このソフトウェアは,ノートPCで5--10Hzの更新速度で動作する.

さらに,カメラとヘッドフォンを一体化し,かつ眼球による情報を遮断するゴーグル型としてハードウェアを設計した.ノートPC上で動作することと併せて,リュックサック一つでスタンドアロンで動作するため,実環境での現実的な使用に耐えるものとなった.

本デバイスを開発者らが数時間連続装用し,「音によって壁の切迫感を感じる」といった主観的な体験を手にいれた.さらに全盲の視覚障害者の方に数十分装着していただき,「(杖を用いた)普段の生活ではわからない情報がわかる」といったような感想を得た.

本デバイスは,選定した深度カメラの性能から,まだ屋内での使用しかできない.さらに,ユーザスタディできた人間の数が限られている.これらの制限はあるものの,上記のことにより,目的としていた「環境に関する情報をセンシングし,音に変換して提示することによって新しい視覚様式を手に入れるデバイス」の開発を実現したと考える.

